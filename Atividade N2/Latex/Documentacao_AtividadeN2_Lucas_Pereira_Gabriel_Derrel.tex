%\documentclass[12pt, openany,oneside, a4paper, oldfontcommands]{abntex2}
%\documentclass[12pt, openright, twoside, a4paper, brazil]{abntex2}
\documentclass[12pt, openany, oneside, a4paper, brazil]{abntex2}
\usepackage[brazil]{babel}
\usepackage[utf8]{inputenc}
\usepackage{indentfirst}
\usepackage{microtype}
%\usepackage{lmodern}
\usepackage{mathptmx} % Define a fonte Times para texto e matemática
\usepackage{amsmath}
\usepackage{amssymb}
\usepackage{graphicx}
\usepackage{minted}
\usepackage{xcolor}
\usepackage{subfig}
\usepackage[num]{abntex2cite}
\usepackage{hyperref}
\usepackage{fancyhdr} % Pacote para cabeçalhos e rodapés personalizados
\usepackage{fvextra}
\DefineVerbatimEnvironment{Highlighting}{Verbatim}{breaklines=true, commandchars=\\\{\}}
\definecolor{bgcolor}{RGB}{45,45,48} 
\usemintedstyle{paraiso-dark}


\begin{document} 
	%\frontmatter
	\begin{center}
		\vspace*{-3cm}
		\includegraphics[width=0.4\textwidth]{puc_go}\\[0.5cm]
		\textbf{PONTIFÍCIA UNIVERSIDADE CATÓLICA DE GOIÁS}\\
		ESCOLA POLITÉCNICA E DE ARTES\\
		ENGENHARIA DE COMPUTAÇÃO\\[2cm]
		
		%\vspace{2cm}
		\MakeUppercase{Lucas Pereira Nunes}\\
		\MakeUppercase{Gabriel Derrel Martins Santee}\\[6cm]
		%\vspace{5cm}
		
		{\ABNTEXchapterfont\bfseries\Large \MakeUppercase{Atividade N2\\Adição de dois números}}\\[3cm]
		
		%\vspace{7.85cm}
		\vfill
		GOIÂNIA\\
		2024
		
	\end{center}
	
	%\newpage
	%\null
	\newpage
	
	\pdfbookmark[0]{\contentsname}{toc}
	\tableofcontents*
	\cleardoublepage
	
	\chapter*[Resposta]{}
	\addtocontents{toc}{\protect\renewcommand{\protect\chapter}[1]{}}
	%\chapter*{}
	\newpage
	\textual
	%\mainmatter
	%\pagenumbering{arabic}  % Define numeração em algarismos arábicos
	\setcounter{page}{1}
	\section*{Código}
	\addcontentsline{toc}{section}{\textbf{1 - Código}}
	Arquivo AtividadeN2\_Main.asm
	\begin{minted}[linenos, breaklines, frame=single, fontsize=\small, tabsize=4, bgcolor=bgcolor]{nasm}
		section .data
		msg1 db "Insira o primeiro numero: "
		msg1_len equ $ - msg1
		msg2 db "Insira o segundo numero: "
		msg2_len equ $ - msg2
		msg_resultado db "Soma: "
		msg_resultado_len equ $ - msg_resultado
		pular_linha_str db 10,10
		pular_linha_len equ $ - pular_linha_str
		buffer1 db 10 ; String do primeiro número
		buffer2 db 10 ; String do segundo número
		buffer_soma db 10 ; String da soma
		buffer_len equ 10 ; Tamanho máximo do buffer
		
		section .bss
		verificar_erro resb 1 ; Byte que verifica se tem algum erro
		
		section .text
		global _start
		extern print_msg, read_msg, atoi, itoa, pularlinha
		global buffer_len, pular_linha_str, pular_linha_len, verificar_erro
		_start:
		; SYSCALL WRITE 1: "Insira o primeiro numero:"
		leitura1:
		mov byte [verificar_erro], 0 ; Zera o verificador de erro
		
		push 4                      ; syscall number para sys_write por pilha
		push 1                      ; file descriptor 1 (stdout) por pilha
		push msg1                   ; endereço da mensagem de prompt por pilha
		push msg1_len               ; tamanho da mensagem de prompt por pilha
		call print_msg              ; chamada da função de impressão de mensagem
		; "Insira o primeiro numero:"
		
		; limpando a pilha
		pop rax
		pop rax
		pop rax
		pop rax
		
		; SYSCALL READ 1
		mov rcx, buffer1            ; aponta para o buffer do primeiro número
		call read_msg               ; chamada da função de leitura da entrada
		
		; Converte string para número inteiro
		mov rsi, buffer1            ; apontando para o início da string do primeiro número
		xor rax, rax                ; zera o valor acumulador(neste caso rax é o acumulador)
		xor rcx, rcx                ; contador de dígitos lidos
		call atoi                   ; chamada da função atoi (ascii para inteiro)
		cmp byte [verificar_erro], 1 ; verifica se tem algum erro
		je leitura1                  ; se tiver erro, volta para leitura1
		push rax             ; armazena o resultado em num1
		
		; SYSCALL WRITE 2: "Insira o segundo numero:"
		leitura2:
		mov byte [verificar_erro], 0 ; Zera o verificador de erro
		
		push 4                      ; syscall number para sys_write por pilha
		push 1                      ; file descriptor 1 (stdout) por pilha
		push msg2                   ; endereço da mensagem de prompt por pilha
		push msg2_len               ; tamanho da mensagem de prompt por pilha
		call print_msg              ; chamada da função de impressão de mensagem
		; "Insira o segundo numero:"
		
		; limpando a pilha
		pop rax
		pop rax
		pop rax
		pop rax
		
		; SYSCALL READ 2
		mov rcx, buffer2            ; aponta para o buffer da entrada
		call read_msg               ; chamada da função de leitura da entrada
		
		; Converte string para número inteiro
		mov rsi, buffer2            ; apontando para o início do buffer
		xor rax, rax                ; zera o valor acumulador
		xor rcx, rcx                ; contador de dígitos lidos
		call atoi                   ; chamada da função atoi (ascii para inteiro)
		cmp byte [verificar_erro], 1 ; verifica se tem algum erro
		je leitura2                  ; se tiver erro, volta para leitura2
		
		; Soma dos dois números
		pop rbx                     ; pegar o valor de num1 da pilha
		add rax, rbx                ; soma os dois valores
		
		; Converte o resultado da soma(que agora está em rax) de volta para string
		mov rbx, rax                ; copia o valor da soma para rbx
		mov rdi, buffer_soma        ; guarda o inicio da string buffer_soma
		add rdi, buffer_len         ; faz o ponteiro rdi apontar para o final da string buffer_soma
		call itoa                   ; chamada da função itoa (inteiro para ascii)
		
		; Exibe "Soma: "
		push 4                  ; syscall number for sys_write
		push 1                  ; file descriptor 1 (stdout)
		push msg_resultado      ; endereço da mensagem de resultado
		push msg_resultado_len  ; tamanho da mensagem de resultado
		call print_msg                    ; chamada de sistema para escrever "Soma: "
		
		; limpando a pilha
		pop rax
		pop rax
		pop rax
		pop rax
		
		; Exibe o número convertido para string
		push 4                  ; syscall number for sys_write
		push 1                  ; file descriptor 1 (stdout)
		push rdi                ; aponta para o início do número no buffer
		push buffer_len         ; tamanho máximo (exibe até o '\0')
		call print_msg                    ; chamada de sistema para escrever o número
		
		; limpando a pilha
		pop rax
		pop rax
		pop rax
		pop rax
		
		; Exibe uma quebra de linha
		call pularlinha
		
		Fim:
		mov rax, 1                  ; Syscall number para sys_exit
		xor rbx, rbx                ; código de saída 0
		int 0x80                    ; chamada de sistema para sair
	\end{minted}
	
	Arquivo AtividadeN2\_Libs.asm
	\begin{minted}[linenos, breaklines, frame=single, fontsize=\small, tabsize=4, bgcolor=bgcolor]{nasm}
		global print_msg, read_msg, atoi, itoa, pularlinha
		extern buffer_len, pular_linha_str, pular_linha_len, verificar_erro
		
		print_msg:
		push rbp                    ; coloca rbp no topo da pilha
		mov rbp, rsp                ; rbp e rsp representam o topo da pilha
		mov rdx, [rbp + 16]          ; acessa 8 bytes abaixo da pilha para acessar o valor pressuposto para rdx
		mov rcx, [rbp + 24]         ; valor pressuposto para rcx
		mov rbx, [rbp + 32]         ; valor pressuposto para rbx
		mov rax, [rbp + 40]         ; valor pressuposto para rax
		int 0x80                    ; chamada de sistema para imprimir
		pop rbp                     ; retira rbp da pilha e rsp volta a ser o unico topo da pilha
		ret
		
		read_msg:
		mov rax, 3                  ; syscall number para sys_read
		mov rbx, 0                  ; file descriptor 0 (stdin)
		mov rdx, buffer_len         ; tamanho do buffer
		int 0x80                    ; chamada de sistema para ler a string
		ret
		
		atoi:
		movzx rbx, byte [rsi]       ; carrega o próximo caractere
		cmp rbx, 10                 ; verifica se é '\0' (final da string)
		je finish_conversion
		; verifica se o caractere é um dígito válido
		cmp rbx, '0'
		jb erro                     ; verifica se é menor que '0'
		cmp rbx, '9'
		ja erro                     ; verifica se é maior que '9'
		sub rbx, '0'                ; converte caractere ASCII para número
		
		imul rax, rax, 10           ; multiplica o valor atual por 10
		add rax, rbx                ; adiciona o dígito convertido
		inc rsi                     ; incrementa para ir para o próximo caractere
		jmp atoi
		; O número convertido está em rax
		finish_conversion:
		ret
		erro:
		mov byte [verificar_erro], 1
		ret
		
		itoa:
		dec rdi                     ; avança para o próximo caractere (de trás para frente)
		mov rax, rbx                ; armazena o valor do quociente(rbx) atual em rax
		mov rdx, 0                  ; zera o registrador rdx, será usado para o resto da divisão
		mov rcx, 10                 ; seta o divisor, ou seja, qual base(10 nesste caso)
		div rcx                     ; divide rax por 10
		add dl, '0'                 ; converte o resto para ASCII
		mov [rdi], dl               ; armazena na string
		mov rbx, rax                ; armazena o quociente para a próxima iteração
		test rbx, rbx               ; verifica se o valor é 0
		jnz itoa
		ret
		
		pularlinha:
		push 4                      ; syscall number para sys_write
		push 1                      ; file descriptor 1 (stdout)
		push pular_linha_str         ; endereço da mensagem de prompt por pilha
		push pular_linha_len         ; tamanho da mensagem de prompt por pilha
		call print_msg              ; chamada da função de impressão de mensagem
		pop rax
		pop rax
		pop rax
		pop rax
		ret
	\end{minted}
	
	\section*{Explicação do Código}
	\addcontentsline{toc}{section}{\textbf{2 - Explicação do Código}}
	
	\subsection*{Arquivo AtividadeN2\_Libs.asm}
	\addcontentsline{toc}{subsection}{\textbf{2.1 - Arquivo AtividadeN2\_Libs.asm}}
	
	O seguinte trecho é responsável por definir quais funções estão no arquivo atual e quais variáveis estão em um arquivo externo.
	\begin{minted}[linenos, breaklines, frame=single, fontsize=\small, tabsize=4, bgcolor=bgcolor]{nasm}
		global print_msg, read_msg, atoi, itoa, pularlinha
		extern buffer_len, pular_linha_str, pular_linha_len, verificar_erro
	\end{minted}
	
	O seguinte trecho define a função responsável por imprimir uma string na tela. Primeiramente colocamos o valor da base da pilha na pilha, isso é necessário pois no passo seguinte o valor do topo da pilha é movido para rsb, isso é feito para termos um ponto fixo onde podemos acessar os valores salvos na pilha.
	
	\noindent Para acessar os valores da pilha fazemos uma aritmética de ponteiro, pegamos da partir de 16 pois queremos os valores a partir da terceira posição, já que a primeira posição guarda o valor da base da pilha e a segunda guarda o endereço de retorno.
	Após passar os valores necessários para os registradores, fazemos a chamada ao sistema, ao chamar o sistema a String será impressa na tela.
	
	\noindent Depois disso, resgatamos o valor da base da pilha de volta para o registrador rbp e damos o retorno.
	
	\begin{minted}[linenos, breaklines, frame=single, fontsize=\small, tabsize=4, bgcolor=bgcolor]{nasm}
		print_msg:
		push rbp                    ; coloca rbp no topo da pilha
		mov rbp, rsp                ; rbp e rsp representam o topo da pilha
		mov rdx, [rbp + 16]          ; acessa 8 bytes abaixo da pilha para acessar o valor pressuposto para rdx
		mov rcx, [rbp + 24]         ; valor pressuposto para rcx
		mov rbx, [rbp + 32]         ; valor pressuposto para rbx
		mov rax, [rbp + 40]         ; valor pressuposto para rax
		int 0x80                    ; chamada de sistema para imprimir
		pop rbp                     ; retira rbp da pilha e rsp volta a ser o unico topo da pilha
		ret
	\end{minted}
	
	O seguinte trecho é responsável por ler uma String do teclado, buffer\_len é uma macro definida em outro arquivo e a string que receberá o que foi lido está foi passada antes de chamar a função. 
	
	\begin{minted}[linenos, breaklines, frame=single, fontsize=\small, tabsize=4, bgcolor=bgcolor]{nasm}
		read_msg:
		mov rax, 3                  ; syscall number para sys_read
		mov rbx, 0                  ; file descriptor 0 (stdin)
		mov rdx, buffer_len         ; tamanho do buffer
		int 0x80                    ; chamada de sistema para ler a string
		ret
	\end{minted}
	
	O seguinte trecho é responsável por converter um número em ASCII para um número inteiro. O registrador rsi guarda o endereço da string a ser convertida, movemos o caractere atual para o registrador rbx utilizando o movzx(nstrução utilizada para mover dados de um registrador para outro registrador, expandindo o valor original com zeros para preenchê-lo totalmente no registrador de destino).
	
	\noindent Agora verificamos se o caractere está entre '0' e '9', se estiver o código continua, senão pula para a label erro, seta a variável de verificação de erro como 1 sai.
	Após passar pela verificação, subtraimos '0' do caractere para obter o seu valor inteiro, após isso multicamos rax por 10(rax é um acumulador), após isso adicionamos ao acumulador o caractere convertido e passamos para o próximo caractere.
	
	\noindent Portanto, digamos que queremos converter a string "123", os valores de rax serão: rax=0, rax=1, rax=10, rax=12, rax=120, rax=123.
	
	\begin{minted}[linenos, breaklines, frame=single, fontsize=\small, tabsize=4, bgcolor=bgcolor]{nasm}
		atoi:
		movzx rbx, byte [rsi]       ; carrega o próximo caractere
		cmp rbx, 10                 ; verifica se é '\0' (final da string)
		je finish_conversion
		; verifica se o caractere é um dígito válido
		cmp rbx, '0'
		jb erro                     ; verifica se é menor que '0'
		cmp rbx, '9'
		ja erro                     ; verifica se é maior que '9'
		sub rbx, '0'                ; converte caractere ASCII para número
		
		imul rax, rax, 10           ; multiplica o valor atual por 10
		add rax, rbx                ; adiciona o dígito convertido
		inc rsi                     ; incrementa para ir para o próximo caractere
		jmp atoi
		; O número convertido está em rax
		finish_conversion:
		ret
		erro:
		mov byte [verificar_erro], 1
		ret
	\end{minted}
	
	O seguinte trecho é responsável por converter um inteiro para uma string. O algorítmo funciona da seguinte forma, rdi guarda a ultima posição da string de destino, isso é necessário pois a inclusão dos caracteres ocorre na ordem inversa, após irmos para a posição anterior, guardamos o quociente anterior(em rbx) em rax, caso seja a primeira iteração o quociente será o inteiro a ser convertido, zeramos o registrador que guardará o resto(rdx) e salvamos a base de converção(nesse caso base 10) em rcx.
	
	\noindent Dividimos rax por rcx e o resultado será guardado em dois registradores, rax(quociente) e rdx(resto), adicionamos '0' ao resto para converte-lo para ASCII e o salvamos o caractere convertido para a posição atual da string; salvamos o quociente em rbx para a próxima iteração. Agora testamos se o quociente é 0, se for a converção terminou, senão a converção continua. 
	
	\begin{minted}[linenos, breaklines, frame=single, fontsize=\small, tabsize=4, bgcolor=bgcolor]{nasm}
		itoa:
		dec rdi                     ; avança para o próximo caractere (de trás para frente)
		mov rax, rbx                ; armazena o valor do quociente(rbx) atual em rax
		mov rdx, 0                  ; zera o registrador rdx, será usado para o resto da divisão
		mov rcx, 10                 ; seta o divisor, ou seja, qual base(10 nesste caso)
		div rcx                     ; divide rax por 10
		add dl, '0'                 ; converte o resto para ASCII
		mov [rdi], dl               ; armazena na string
		mov rbx, rax                ; armazena o quociente para a próxima iteração
		test rbx, rbx               ; verifica se o valor é 0
		jnz itoa
		ret
	\end{minted}
	
	O seguinte trecho é responsável pela quebra de linha, para isso imprimimos a string pular\_linha\_str passando os valores pela pilha e chamando a função se imprimir uma mensagem na tela.
	
	\begin{minted}[linenos, breaklines, frame=single, fontsize=\small, tabsize=4, bgcolor=bgcolor]{nasm}
		pularlinha:
		push 4                      ; syscall number para sys_write
		push 1                      ; file descriptor 1 (stdout)
		push pular_linha_str         ; endereço da mensagem de prompt por pilha
		push pular_linha_len         ; tamanho da mensagem de prompt por pilha
		call print_msg              ; chamada da função de impressão de mensagem
		pop rax
		pop rax
		pop rax
		pop rax
		ret
	\end{minted}
	
	\subsection*{Arquivo AtividadeN2\_Main.asm}
	\addcontentsline{toc}{subsection}{\textbf{2.2 - Arquivo AtividadeN2\_Main.asm}}
	
	O Seguinte trecho é responsável por definir as variáveis e definir quais são as variáveis globais e quais são as funções externas.
	
	\begin{minted}[linenos, breaklines, frame=single, fontsize=\small, tabsize=4, bgcolor=bgcolor]{nasm}
		section .data
		msg1 db "Insira o primeiro numero: "
		msg1_len equ $ - msg1
		msg2 db "Insira o segundo numero: "
		msg2_len equ $ - msg2
		msg_resultado db "Soma: "
		msg_resultado_len equ $ - msg_resultado
		pular_linha_str db 10,10
		pular_linha_len equ $ - pular_linha_str
		buffer1 db 10 ; String do primeiro número
		buffer2 db 10 ; String do segundo número
		buffer_soma db 10 ; String da soma
		buffer_len equ 10 ; Tamanho máximo do buffer
		
		section .bss
		verificar_erro resb 1 ; Byte que verifica se tem algum erro
		
		section .text
		global _start
		extern print_msg, read_msg, atoi, itoa, pularlinha
		global buffer_len, pular_linha_str, pular_linha_len, verificar_erro
	\end{minted}
	
	No seguinte trecho, o comportamento de leitura1 e de leitura2 é identico, zerando o verificador de erro, imprimindo msg1 pela função externa print\_msg, lendo uma String do usuário pela função externa read\_msg passando a string que guardará o que foi digitado para a função pelo registrador rcx, e convertendo a string lida chamando a função externa atoi.
	
	\noindent Caso haja um erro, a lógica de leitura será recomaçada, ou seja, caso em leitura1 a string seja inválida, o primeiro número será pedido de novo.
	
	\begin{minted}[linenos, breaklines, frame=single, fontsize=\small, tabsize=4, bgcolor=bgcolor]{nasm}
		; SYSCALL WRITE 1: "Insira o primeiro numero:"
		leitura1:
		mov byte [verificar_erro], 0 ; Zera o verificador de erro
		
		push 4                      ; syscall number para sys_write por pilha
		push 1                      ; file descriptor 1 (stdout) por pilha
		push msg1                   ; endereço da mensagem de prompt por pilha
		push msg1_len               ; tamanho da mensagem de prompt por pilha
		call print_msg              ; chamada da função de impressão de mensagem
		; "Insira o primeiro numero:"
		
		; limpando a pilha
		pop rax
		pop rax
		pop rax
		pop rax
		
		; SYSCALL READ 1
		mov rcx, buffer1            ; aponta para o buffer do primeiro número
		call read_msg               ; chamada da função de leitura da entrada
		
		; Converte string para número inteiro
		mov rsi, buffer1            ; apontando para o início da string do primeiro número
		xor rax, rax                ; zera o valor acumulador(neste caso rax é o acumulador)
		xor rcx, rcx                ; contador de dígitos lidos
		call atoi                   ; chamada da função atoi (ascii para inteiro)
		cmp byte [verificar_erro], 1 ; verifica se tem algum erro
		je leitura1                  ; se tiver erro, volta para leitura1
		push rax             ; armazena o resultado em num1
		
		; SYSCALL WRITE 2: "Insira o segundo numero:"
		leitura2:
		mov byte [verificar_erro], 0 ; Zera o verificador de erro
		
		push 4                      ; syscall number para sys_write por pilha
		push 1                      ; file descriptor 1 (stdout) por pilha
		push msg2                   ; endereço da mensagem de prompt por pilha
		push msg2_len               ; tamanho da mensagem de prompt por pilha
		call print_msg              ; chamada da função de impressão de mensagem
		; "Insira o segundo numero:"
		
		; limpando a pilha
		pop rax
		pop rax
		pop rax
		pop rax
		
		; SYSCALL READ 2
		mov rcx, buffer2            ; aponta para o buffer da entrada
		call read_msg               ; chamada da função de leitura da entrada
		
		; Converte string para número inteiro
		mov rsi, buffer2            ; apontando para o início do buffer
		xor rax, rax                ; zera o valor acumulador
		xor rcx, rcx                ; contador de dígitos lidos
		call atoi                   ; chamada da função atoi (ascii para inteiro)
		cmp byte [verificar_erro], 1 ; verifica se tem algum erro
		je leitura2                  ; se tiver erro, volta para leitura2
	\end{minted}
	
	O seguinte trecho é responsável por somar os 2 números e converter a soma para string chamando a função externa itoa.
	
	\begin{minted}[linenos, breaklines, frame=single, fontsize=\small, tabsize=4, bgcolor=bgcolor]{nasm}
		; Soma dos dois números
		pop rbx                     ; pegar o valor de num1 da pilha
		add rax, rbx                ; soma os dois valores
		
		; Converte o resultado da soma(que agora está em rax) de volta para string
		mov rbx, rax                ; copia o valor da soma para rbx
		mov rdi, buffer_soma        ; guarda o inicio da string buffer_soma
		add rdi, buffer_len         ; faz o ponteiro rdi apontar para o final da string buffer_soma
		call itoa                   ; chamada da função itoa (inteiro para ascii)
	\end{minted}
	
	O seguinte trecho imprime a string convertida e encerra o programa. O registrador rdi é passado ao invés de buffer\_soma pois não necessáriamente o inicio da variável é o inicío da string so número convertido, visto que a inclusão caracteres em buffer\_soma ocorre de trás pra frente.
	
	\begin{minted}[linenos, breaklines, frame=single, fontsize=\small, tabsize=4, bgcolor=bgcolor]{nasm}
		; Exibe "Soma: "
		push 4                  ; syscall number for sys_write
		push 1                  ; file descriptor 1 (stdout)
		push msg_resultado      ; endereço da mensagem de resultado
		push msg_resultado_len  ; tamanho da mensagem de resultado
		call print_msg                    ; chamada de sistema para escrever "Soma: "
		
		; limpando a pilha
		pop rax
		pop rax
		pop rax
		pop rax
		
		; Exibe o número convertido para string
		push 4                  ; syscall number for sys_write
		push 1                  ; file descriptor 1 (stdout)
		push rdi                ; aponta para o início do número no buffer
		push buffer_len         ; tamanho máximo (exibe até o '\0')
		call print_msg                    ; chamada de sistema para escrever o número
		
		; limpando a pilha
		pop rax
		pop rax
		pop rax
		pop rax
		
		; Exibe uma quebra de linha
		call pularlinha
		
		Fim:
		mov rax, 1                  ; Syscall number para sys_exit
		xor rbx, rbx                ; código de saída 0
		int 0x80                    ; chamada de sistema para sair
	\end{minted}
	
\end{document}